\documentclass{beamer}
\usepackage{tikz}
\usepackage[T2A]{fontenc}
\usepackage[all]{xy}
\usepackage[utf8]{inputenc}
\usepackage[russian]{babel}
\usepackage{hyphenat}
\usepackage[T2A,T1]{fontenc}
\usepackage{amsmath}
\usepackage{graphicx}
\usepackage{listings}
\usepackage{xcolor}

\usetheme{Madrid}

\title{Сети и потоки. Алгоритм Диница}

\author{~Кононов Николай}

\institute[СПБГУ]
{
    Математико-Механический факультет СПбГУ
}
\date{2019}

\AtBeginSubsection[]
{
  \begin{frame}<beamer>{План}
    \tableofcontents[currentsection,currentsubsection]
  \end{frame}
}

\begin{document}

\begin{frame}
  \titlepage
\end{frame}

\begin{frame}{План}
  \tableofcontents
\end{frame}


\section{Сети и потоки}

\subsection{Простейшие понятия}

\begin{frame}{Сеть}
  \begin{itemize}
  \item {
    \textbf{Определение: } Пусть есть множество вершин V, в котором выделены две вершины: s (вход или исток) и t (выход или сток). \\Пусть определена функция $c:V \times V \rightarrow \mathbb{R}$, удовлетворяющая соотношениям $$\forall x, y \in V \quad c(x, y) \geq 0, \quad c(x, s) = 0, \quad c(t, y) = 0 $$
    функция $c$ - \textbf{пропускная способность}.
    \pause
  }
  \item {
    $A = \{(x, y) : c(x, y) > 0\}$ - множество стрелок\\  Тогда $ G = ((V, A), s, t, c) $ - \textbf{сеть}
  }
  \end{itemize}
\end{frame}

% You can reveal the parts of a slide one at a time
% with the \pause command:
\begin{frame}{Поток в сети}
  \begin{itemize}
  \item {
    \textbf{Определение: } Пусть $G - $ сеть, а функция $f:V \times V \rightarrow \mathbb{R}$ удовлетворяет трем условиям:\\
        1) $\forall x, y \in V \quad f(x, y) \leq c(x, y)$\\
        2) $\forall x, y \in V \quad f(x, y) = -f(y, x)$\\
        3) $\forall v \in V\textbackslash \{s, t\}$ выполняется условие: $ \sum\limits_{x \in V} f(v, x) = 0$ - закон сохранения потока
    \\ $f$ - \textbf{поток в сети $G$}
    \pause
  }
  \item {
    $ |f| = \sum\limits_{v \in V} f(s, v)$ - величина потока\\
        Поток с максимальной величиной - \textbf{максимальный}
  }
  \end{itemize}
\end{frame}

\begin{frame}{Разрез сети}
  \begin{itemize}
    \item {
        \textbf{Определение: } пусть $G$ - сеть, а множество ее вершин $V$ разбито на два дизъюнктых множества $S \ni s$ и $T \ni t$. Тогда $(S,T)$ - \textbf{разрез сети G}
        \pause
    }
    \item {
        Величина $c(S, T) =  \sum\limits_{x \in S, y \in T}c(x, y)$ называется \textbf{пропускной способностью разреза}. \\ Любой разрез сети G с минимальной пропускной способностью называется \textbf{минимальным}.
        \pause
    }
    \item {
        Для любого потока f величина $f(S, T) =  \sum\limits_{x \in S, y \in T} f(x, y)$ называется \textbf{потоком через разрез}.
        \pause
    }
    \end{itemize}
    \begin{block}{Лемма}
        \textbf{Лемма: } Для любого потока $f$, и разреза $(S, T)$ сети $G$ выполняется $|f| = f(S, T)$
    \end{block}
\end{frame}

\subsection{Остаточная сеть, блокирующий поток}

\begin{frame}{Остаточная сеть}
    \begin{itemize}
    \item {
        \textbf{Остаточной пропускной способностью} $c_{f}$ по отношению к сети $G = \{(V, E), s, t, c\}$ и потоку $f$ в ней называется пропускная способность
        \begin{equation*}
        c_{f}(x, y) = 
         \begin{cases}
           0 &\text{if y = s or x = t}\\
           c(x, y) - f(x, y) &\text{otherwise}
         \end{cases}
        \end{equation*}
        \\
        \pause
    }
    \item {
        \textbf{Остаточной сетью} для сети $G$ и потока $f$ называется сеть $G_{f} = \{(V, E_{f}), s, t, c_{f}\}$, где $E_{f} = \{ (u, v) \in V \times V | c_{f}(u, v) > 0\}$
    }
    \item {
        Остаточное ребро можно интуитивно понимать как меру того, насколько можно еще увеличить поток вдоль этого ребра
        \pause
    }
    \begin{exampleblock}{Определение}
    Простой st-путь в $G_{f}$ называется \textbf{дополняющим путем}
    \end{exampleblock}

    \end{itemize}
\end{frame}

\begin{frame}{Блокирующий поток}
    \small {
    \begin{exampleblock}{Определение}
        \textbf{Блокирующим потоком $f$} в сети $G = ((V, E), s, t, c)$ называется такой поток, что $\forall st$-путь содержит насыщенное этим потоком ребро. То есть в данной сети не найдется такого пути из истока в сток, вдоль которого можно безпрепятственно увеличить поток 
    \end{exampleblock}
    \textbf{Замечание: } блокирующий поток не всегда максимальный, более того, он может быть сколь угодно малым, относительно максимального \\
    \pause
    \textbf{Пример:} пропускная способность ребер 1, 
        'единичный' поток идет по красным ребрам
    }
    \begin{figure}[h!]
        \advance\leftskip-3cm
        \advance\rightskip-3cm

        \includegraphics[scale=0.7]{blocking_flow.png}
        \label{fig: blocking_flow}
    \end{figure}
\end{frame}

\subsection{Теорема Форда-Фалкерсона}
\begin{frame}{Теорема Форда-Фалкерсона}
    \begin{theorem}
        \textbf{Ford-Fulkerson:} В сети $G$ с пропускной способностью $c$ задан поток $f$, тогда \textit{следующие три утверждения равносильны:\\
        1) Поток f максимален\\
        2) $\exists (S, T)$ : $|f| = c(S, T)$\\
        3) В остаточной сети $G_f$ нет дополняющего пути
        }
    \end{theorem}
\end{frame}

\subsection{Слоистая сеть}

\begin{frame}{Слоистая сеть}
    \textbf{Слоистая сеть(layered network, вспомогательная сеть)} строится след образом:
    \begin{itemize}
        \item {
            Для каждой вершины $v$ данной сети $G$ определим длину кратчайшего $s \rightsquigearrow v$-пути из истока и обозначим ее \texttt{d[v]} (можно сделать обходом в ширину)
        }
        \item {
            В слоистую сеть включаем только стрелки $(u, v)$ такие, что \texttt{d[u] = d[v] + 1}
        }
        \item {
            То есть исключим из $G$ стрелки лежащие внутри одного уровня или идущие назад
        }
        \item {
            Получившаяся сеть ациклична и любой $s \rightsquigarrow t$ путь в слоистой сети является кратчайшим путем в исходной сети из свойств \textbf{BFS}
        }
        \item {
         $ G = \{\{1, 2\}, \{3\}, \{4\}, \{2, 5\}, \{3, 5\}, \{1, 6\}\}; \quad s = 0, t = 6$
         \\тогда \textbf{слоистая сеть $G_{s}$} = $\{\{1, 2\}, \{3\}, \{4\}, \{5\}, \{5\}, \{6\}\}$
        }
    \end{itemize}
\end{frame}

\begin{frame}{Пример слоистой сети}
    \begin{figure}[h!]
        \centering
        \includegraphics[scale=0.4]{network.png}
        \label{fig: blocking_flow}
    \end{figure}
\end{frame}

\section{Алгоритм Диница}

\subsection{Основные идеи}

\begin{frame}{Алгоритм Диница. Основные идеи}
    \begin{block}{Постановка задачи}
    Пусть дана сеть $G = ((V, E), s, t, c)$. Как найти поток $f$ из $s$ в $t$ максимальной величины?
    \end{block}
    \begin{itemize}
        \item {
        Алгоритм является улучшенной версией \textbf{Алгоритма Эдмонса-Карпа}
        \pause
        }
        \item {
        Изначально пусть $f(e) = 0 \quad \forall e \in E$
        \pause
        }
        \item {
        Алгоритм состоит из нескольких \textbf{фаз}.
        \pause
        }
        \item {
        На каждой фазе строится остаточная сеть \textbf{$G_f$}, затем по отношению к $G_f$ строится слоистая сеть $G_{L}$(\textbf{BFS}). 
        
        Если \texttt{d[t] = $\infty$} останавливаемся и выводим $f$
        \pause
        }
        \item {
        В построенной слоистой сети находим блокирующий поток $f`$ (любой)
        \pause
        }
        \item {
        Дополняем поток $f$ потоком $f`$ и переходим к следующей фазе
        }
    \end{itemize}
\end{frame}

\subsection{Пример}
\begin{frame}{Пример}
    \begin{itemize}
    \item {
        f = 0
    }
    \item {
            $(G, f) \rightarrow G_{f} \rightarrow G_{L} \rightarrow f' \rightarrow f = f + f`$
        }
    \end{itemize}

    \begin{figure}[h!]
        \centering
        \includegraphics[scale=0.15]{Dinic_algorithm_G1.png}
        \label{fig: example}
    \end{figure}
\end{frame}

\subsection{Корректность}
\begin{frame}{Корректность алгоритма}
    \begin{theorem}
        Если алгоритм завершается, полученный поток является потоком максимальной длины.
    \end{theorem}
    \pause
    \begin{proof}
        Предположим, что в какой-то момент в слоистой сети $G_{L}$ построенной для остаточной сети $G_f$ не удалось найти блокирующий поток.\\
        \pause
        Это означает, что $\texttt{d[t] = } \infty$, то есть сток $t$ не достижим из истока $s$ в слоистой сети .\\
        \pause
        Но слоистая сеть содержит в себе все кратчайшие пути в сети $G_{f}$ из истока $s$.\\
        \pause
        Таким образом в остаточной сети нет 
        $s \rightsquigarrow t$ пути\\
        \pause
        Применяя теорему Форда-Фалкерсона получаем, что текущий поток в самом деле максимален.
    \end{proof}
\end{frame}

\subsection{Асимптотика и оценка числа фаз}

\begin{frame}{Оценка числа фаз}{Lemma 1}
\begin{lemma}
        Кратчайшее расстояние между истоком и стоком устрого увеличивается с выполнением каждой итерации:
        \texttt{$d_{i}[t] > d_{i-1}[t] \quad \forall i$}
    \end{lemma}
    
    \begin{proof}
        От противного. Пусть длина кратчайшего $s \rightsquigarrow t$ пути не изменилась после $i$-ой итерации.
        Слоистая сеть $G_{L}$ строится по остаточной $G_{f}$. Рассмотрим кратчайший $s \rightsquigarrow t$ путь. По предположению его длина должна остаться неизменной. Однако $G^{i}_{f}$ содержит только ребра остаточной сети перед i-й фазой, либо обратные к ним. \\
        \pause
        Таким образом пришли к противоречию: нашелся $s \rightsquigarrow t$ путь, который не содержит насыщенных ребер, и имеет ту же длину, что и кратчайший путь. Этот путь должен был быть "заблокирован" блокирующим потоком.
    \end{proof}
\end{frame}

\begin{frame}{Оценка числа фаз}{Lemma 2}
    \begin{lemma}
        Кратчайшее расстояние от истока до каждой вершины не уменьшается с выполнением каждой итерации:
        \forall v \in V \quad \texttt{$d_{i}[v] \geq d_{i-1}[v] $}
    \end{lemma}
    \begin{proof}
        Расмотрим произвольные $v$ и $i$ и кратчайший $s \rightsquigarrow v$-путь P в сети $G^{i}_{f}$.
        $|P| = d_{i}[v]$
        Заметим, что в остаточную сеть $G^{i}_{f}$ могут входить стрелки $G_{f}$, а также стрелки обратные к ним.
        \pause
        Рассмотрим 2 случая:
        \pause
        \begin{itemize}
            \item Путь P содержит только ребра из $G_f$. Тогда $|P| \geq d_{i}[v]$ ($d_i[v]$ - длина кратчайшего пути) $\Longleftrightarrow d_i[v] \geq d_{i-1}[v]$
        \end{itemize}
    \end{proof}
\end{frame}

\begin{frame}{Окончание доказательства}
    \begin{proof}
    \begin{itemize}
        \item Путь P содержит хотя бы одно ребро, не содержащееся в сети $G_{f}$, но обратное какому-то из ее ребер. Рассмотрим первое такое ребро $(u, w)$ в пути P:
        $s \Longrightarrow u \rightarrow v \Longrightarrow t$
        Применим лемму к вершине u, т.к. она удовлетворяет условию первого случая: $d_{i}[u] \geq d_{i-1}[u]$(1)
        \pause
        Теперь заметим, что т.к. (u, w) появилось в остаточной сети только после выполнения (i-1)-ой фазы $\Rightarrow$ вдоль ребра (w, u) был дополнительно пропущен какой-то поток. Следовательно, ребро (w, u) пренадлежало слоистой сети перед (i-1)-й фазой \Rightarrow $d_{i-1}[u] = d_{i-1}[v] + 1$(2) \\
        По свойству кратчайших путей: $d_{i}[w] = d_{i}[u] + 1$(3)
        Объединяя (1), (2), (3) получим: $d_{i}[w] \geq d_{i-1}[w] + 2$. Теперь мы можем применять те же рассуждения ко всему оставшемуся пути до v и получить требуемое неравенство
    \end{itemize}
    \end{proof}
\end{frame}

\begin{frame}{Оценка числа фаз}
    \begin{itemize}
    \item {
        Так как длина кратчайшего $s \rightsquigarrow t$ пути не может превосходить $n - 1 \Rightarrow$ алгоритм Диница совершает не больше $n - 1$ фазы (итераций цикла).
        \pause
    }
    \item {
        Таким образом, в зависимости от того, каким алгоритмом нахождения блокирующего потока мы пользовались алгоритм Диница может выполнятся за $O(|V| \cdot |E|^{2})$ или за $O(|V|^{2} \cdot |E|)$\\
        \pause
    }
    \item {
        Возможно достичь асимптотики $O(|V| \cdot |E| \cdot log(|V|)$, используя \textbf{динамические деревья Слетора и Тарьяна}
    }

    \end{itemize}
\end{frame}

\subsection{Поиск блокирующего потока}
\begin{frame}{Поиск блокирующего потока}{Жадный алгоритм}
    \begin{itemize}
        \item {
            Так как слоистая сеть $G_{L}$, в которой ищется блокирующий поток ациклическая - будем искать блокирующий поток в ациклической сети.
        }
        \item Искать $s \rightsquigarrow t$ пути по одному, пока такие пути находятся
        \item DFS найдет все $s \rightsquigarrow t$ пути, если t достижима из s, а $c(u, v) > 0 \quad \forall (u, v) \in E$
        \item { Насыщая ребра, мы хотя бы единожды достигнем стока t, следовательно блокирующий поток всегда найдется. 
        \item DFS находит каждый путь за O(E), каждый путь насыщает как минимум одно ребро \Rightarrow всего будет O(E) путей\\
        Итоговая асимптотика: O($E^2$)
        }
    \end{itemize}
\end {frame}

\begin{frame}{Оптимизация. Удаляющий обход}
    \begin{itemize}
        \item Будем использовать предыдущий алгоритм, удаляя при этом ребра, из которых невозможно дойти до стока t
        \pause
        \item Достаточно удалять ребро после того, как мы просмотрели его в DFS, если не нашелся путь до стока
        \pause
        \item Будем поддерживать в списке смежности каждой вершины указатель на первое удаленное ребро и увеличивать его внутри цикла DFS
        \pause
        \item Если DFS достигает стока: насыщается как минимум одно ребро. Иначе как минимум один указатель продвигается вперед. Значит один запуск обхода в глубину работает за O(V + K), K - число продвижения указателей. 
        Всего запусков DFS для поиска блокирующего потока: O(P), где P - количество ребер, насыщенных блокирующим потоком.
        Таким образом весь алгоритм отработает за $O(P \cdot V + \sum_{i} K_{i} $ = $O(P \cdot V + E)$. В худшем случае, когда P = E, $O(V \cdot E)$
    \end{itemize}
\end{frame}
\begin{enumerate}
\end{enumerate}

\subsection{Реализация}
\begin{frame}{Реализация}{Удаляющий обход}

\end{frame}

\begin{frame}[allowframebreaks]
  \frametitle<presentation>{Литература}

  \begin{thebibliography}{10}

  \beamertemplatebookbibitems

  \bibitem{Кормен2007}
    Т.~Кормен.
    \newblock {\em Алгоритмы. Построение и анализ}.
    \newblock Глава 27, "Максимальный поток".


  \beamertemplatearticlebibitems

  \bibitem{}
      ~neerc.ifmo.ru
      \newblock "Схема алгоритма Диница"
  \end{thebibliography}
\end{frame}

\end{document}
